\section{Introduction}

Over the past decades, email has gained wide-spread adoption 
and is now a core communication service for society. However, 
some of its original design choices have had a profound impact 
on email information security and storage, making it a fundamentally
insecure communication channel. In this paper, we revisit some 
of these choices and propose a backwards-compatible email service, 
called STEAK, that addresses this limitation in the context of 
webmail-like usability expectations.

Users send a wide variety of information in their email messages, 
including financial documents, healthcare records, legal documents, etc.
Neither traditional email protocols, like SMTP and IMAP, nor contemporary
webmail, provide adequate security guarantees.
In a secure communication channel, when Alice sends 
Bob a message, only Bob can read it (message confidentiality), Bob can 
verify that the message he received was in fact sent by Alice (message 
authenticity), and that it contains the data she sent (message integrity). 
Email and webmail provide none of these properties out of the box. 

At the same time, users have come to expect certain features on top of 
traditional email.  These include automatic spam filtering, the ability 
to search and organize messages, and ubiquitous access to their email 
through webmail from a variety of user devices. We believe that providing 
stronger security guarantees cannot come at an expense of reduced usability. 
Our goal is to design an email service that provides these properties 
(called CIA guarantees in the rest of the paper), while providing the 
features and user experience of webmail.  This is a nontrivial problem 
and current security approaches require running out-of-band security 
systems ``on top'' of email usually by leveraging public-key cryptography. 
These include S/MIME and PGP, as well as more recent ID-based encryption 
schemes~\cite{id-based-cryptography}.

The problem with out-of-band security approaches is that they require
 active involvement in key management. 
Users must generate, distribute, and revoke public keys with the out-of-band 
system, and carefully guard their private keys while remaining vigilant 
for compromises. We believe this is unreasonable because most users do 
not understand practical information security~\cite{garfinkel-email-survey}, 
and want the convenience 
of webmail despite the security problems it introduces. Even if they 
understood the security concerns, using public-key cryptography in this 
manner greatly increases the complexity of basic email tasks. 

Our key insight is that email's store-and-forward approach makes CIA 
guarantees hard to achieve. Because each email server stores and processes 
messages, a user must either trust the server completely to not to break 
CIA, or perform end-to-end authenticated encryption outside of the system. 
The former is unrealistic, but the latter requires users to set up and 
manage keys out-of-band. To address these challenges in this paper we 
present a new email system called Security Transparent Email with 
Automatically-managed Keys (STEAK). 

The contributions of STEAK are 1) an automatic key management system 
(called AutoKey) and 2) an email exchange protocol (called Secure Message 
Request Protocol, or SMRP) that allow users to access their email using a 
web browser on any device. STEAK enables access to email with only a 
username/password pair and a small amount of additional logic at the 
email client. To do so, STEAK leverages the user's own personal cloud 
storage to host sealed account state, sealed keys, and sealed messages. 
The user devices (clients), not servers, process messages after decrypting 
them.  The AutoKey system distributes public keys using more non-colluding 
key repositories than an adversary can compromise.  Once keys are distributed, users
fetch mail with SMRP by downloading them from the sender's cloud storage. All the while, 
STEAK remains transparently backwards-compatible with SMTP, and offers 
more limited CIA guarantees to non-STEAK users that are still stronger than SMTP.

The remainder of this paper is organized as follows.  In section~\ref{sec:motivation}, 
we define our usability requirements, and argue that our strategy is necessary 
to meet them for providing CIA guarantees. Then, we present the design of STEAK 
in section~\ref{sec:design}, with a focus on AutoKey and SMRP, and describe how basic 
email activities are performed.  Afterward, we present our 
implementation strategies and prototype (section~\ref{sec:implementation}), and give 
a qualitative usability 
analysis and a preliminary performance evaluation (section~\ref{sec:evaluation}).  
We finish with related work (Section~\ref{sec:related-work}) and a conclusion (section~\ref{sec:conclusion}).
